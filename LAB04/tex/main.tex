% example.tex - Minimal but feature-rich LaTeX example
% Compile with: pdflatex example.tex && bibtex example (or use latexmk -pdf example.tex)

\documentclass[11pt,a4paper]{article}

% Encoding and font/typography
\usepackage[T1]{fontenc}
\usepackage[utf8]{inputenc}
\usepackage{microtype}

% Page layout
\usepackage[margin=2.5cm]{geometry}

% Maths, figures, tables, code
\usepackage{amsmath,amssymb}
\usepackage{graphicx}
\usepackage{booktabs}
\usepackage{caption}
\usepackage{listings}
\usepackage{color}
\usepackage{hyperref}
\usepackage{subcaption}   % w preambule

% Listings (code) style
\definecolor{codegray}{rgb}{0.95,0.95,0.95}
\lstset{
  backgroundcolor=\color{codegray},
  basicstyle=\ttfamily\small,
  frame=single,
  breaklines=true,
  captionpos=b
}

% Document metadata
\title{PRiW Zadanie 4}
\author{Kacper Małecki}

\begin{document}
\maketitle
\begin{abstract}
    Sprawozdania, które omawia użycie OpenMP, wyniki czasowe oraz sposoby podziału zadania dla wątku
    przy wielowątkowym obliczaniu fraktala Mandelbrota.
\end{abstract}


\section{Sposoby podziału zadań}
Przy wykonaniu zadania posłużyłem się 3 metodami (harmonogramami) podziału zadań dla wątków.
\subsection{Static}
Harmonogram static, który dzieli zadanie na prawie równe bloki zgodnie z ilością, którą
mu podamy; gdy tego nie zrobimy każdy wątek dostanie jedną spójną część bloku. \break
\begin{figure}[htbp]
    \centering
    
    \begin{subfigure}{0.25\textwidth}
        \centering
        \includegraphics[width=\textwidth]{../PPMs/Static16.png}
        \caption{Obraz 1}
    \end{subfigure}
    \begin{subfigure}{0.25\textwidth}
        \centering
        \includegraphics[width=\textwidth]{../PPMs/Static16.png}
        \caption{Obraz 2}
    \end{subfigure}

    \vspace{1em}

    \begin{subfigure}{0.25\textwidth}
        \centering
        \includegraphics[width=\textwidth]{../PPMs/Static16.png}
        \caption{Obraz 3}
    \end{subfigure}
    \begin{subfigure}{0.25\textwidth}
        \centering
        \includegraphics[width=\textwidth]{../PPMs/Static16.png}
        \caption{Obraz 4}
    \end{subfigure}

    \caption{Podział bloków w zależności od ilości użytych wątków}
\end{figure}

\subsection{Dynamic}

\subsection{Guided}


\section{Wyniki czasowe}
Poniżej wykres który przedstawia zależność czasu wykonania od ilości użytych wątków:
\begin{figure}[htbp]
    \centering
        \includegraphics[width=0.6\textwidth]{../Plots/mandelbrot_size_1000.png}
    \caption{}\label{}
\end{figure}


\end{document}
